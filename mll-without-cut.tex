\section{Correctness for $\mathsf{MLL+Mix}$ without Cut}
In this section, we approach the classical problem of linear logic---namely, the correctness of proof structures---from the perspective of creed and kit, providing a new viewpoint on the relevance of these concepts.


\subsection{Preliminaries: MLL proof structures and proof-nets}

\newcommand{\PropVar}{\alpha}
\newcommand{\PropVarb}{\beta}
\newcommand{\PropVarc}{\gamma}

A \emph{formula} is given by the following grammar:
\begin{equation*}
    A,B,C
    \quad::=\quad
    \PropVar \mid \PropVar^{\bot} \mid A \otimes B \mid A \llpar B,
\end{equation*}
where \( \PropVar \) is a propositional variable and \( \PropVar^{\bot} \) is its negation.
A \emph{literal} is a propositional \( \PropVar \) or its negation \( \PropVar^{\bot} \).
For simplicity, we consider only formulas in negation normal form (i.e.~the negation \( ({-})^{\bot} \) is applicable to only propositional variables).
The operation \( ({-})^{\bot} \) on formulas is defined as usual.
The tensor unit is abscent as in many other papers on this subject.\tk{cite?}

A \emph{squent} \( \Gamma \) is a finite sequence of formulas.
The concatenation of sequents \( \Gamma \) and \( \Delta \) is written as \( \Gamma, \Delta \).
The (one-sided) sequent calculus \( \mathsf{MLL} \) is defined by the rules in \cref{fig:sequent-calculus-mll}.
The sequent calculus \( \mathsf{MLL+Mix} \) is defined by the rules in \cref{fig:sequent-calculus-mll} with the \( \mathsf{Mix} \) rule in \cref{fig:mix-rule}.
\begin{figure}[t]
    \begin{gather*}
        \dfrac{\mathstrut}{\vdash A, A^\bot}(\mathsf{Ax})
        \quad
        \dfrac{\vdash \Gamma, A \qquad \vdash \Delta, B}{\vdash \Gamma, \Delta, A \otimes B}(\otimes)
        \quad
        \dfrac{\vdash \Gamma, A, B}{\vdash \Gamma, A \llpar B}(\llpar)
        \\[5pt]
        \dfrac{\vdash \Gamma, A, B, \Delta}{\vdash \Gamma, B, A, \Delta}(\mathsf{Ex})
        \quad
        \dfrac{\vdash \Gamma, A \qquad \vdash \Delta, A^\bot}{\vdash \Gamma, \Delta}(\mathsf{Cut}).
    \end{gather*}
    \vspace{-3ex}
    \caption{The proof rules of the sequent calculus \( \mathsf{MLL} \).}
    \label{fig:sequent-calculus-mll}
\end{figure}
\begin{figure}[t]
    \begin{equation*}
        \dfrac{\vdash \Gamma \qquad \vdash \Delta}{\vdash \Gamma, \Delta}(\mathsf{Mix}).
    \end{equation*}
    \vspace{-3ex}
    \caption{The rule \( \mathsf{Mix} \).}
    \label{fig:mix-rule}
\end{figure}
A proof is \emph{cut-free} if it does not use the \( \mathsf{Cut} \) rule.

A \emph{proof structe} \( \pi \) is a triple \( (\Gamma, \Delta, \Phi) \) where \( \Gamma \) is a sequent, \( \Delta \) is a sequent of the form \( A_1, A_1^{\bot}, A_2, A_2^{\bot}, \dots, A_n, A_n^{\bot} \) and \( \Phi \) is a \emph{linking} on \( \Gamma,\Delta \), which is a bijection on occurrences of literals in \( \Gamma,\Delta \) such that (1) \( \Phi \) maps an occurrence of \( \PropVar \) to an occurrence of \( \PropVar^{\bot} \) and (2) \( \Phi \circ \Phi = \ident \).
A \emph{cut-free proof structure} is a proof structure with the empty \( \Delta \).
For \( \pi = (\Gamma, \Delta, \Phi) \), we write \( \pi \vdash \Gamma \) and call it a proof structure of \( \Gamma \).
\tk{A graphical presentation of a proof structure.}

A sequent calculus proof induces a proof structure.
\tk{to do: define proof structures; define the structure associated to a proof; define the sequentilisability}

However, the converse does not hold: there exists a proof structure that cannot be obtained from a sequent calculus proof.
A proof structure is \emph{sequentialisable} or \emph{correct} if it comes from a sequent calculus proof.
An important problem here is to characterise sequentialisable proof structures.
Such a condition is called a \emph{correctness criterion} and many correctness criteria have been known~\tk{to do: citation}.

\emph{Danos-Regnier criterion} is a famous correctness criterion.\tk{citation}
A \emph{Danos-Regnier switching} is a choice of left/right for each occurrence of \( \llpar \).
The graph associated to a Danos-Regnier switching is obtained by removing every edge between a \( \llpar \)-node and its unchosen child.
\begin{theorem}[Danos and Regnier~\cite{Danos1989}]\label{thm:danos-regnier}
    A proof structure is correct with respect to \( \mathsf{MLL} \) if and only if, for every Danos-Regnier switching, the associated graph is a tree.
\end{theorem}
\begin{theorem}[\tk{todo: citation}]\label{thm:weak-danos-regnier}
    A proof structure is correct with respect to \( \mathsf{MLL+Mix} \) if and only if, for every Danos-Regnier switching, the associate graph is acyclic.
\end{theorem}
\tk{to do: add examples}


\subsection{Semantics of Proof Structures}
We define the interpretation of a proof structure.
A standard model of \( \mathsf{MLL} \) is a \( * \)-autonomous category, but this structure is insufficient to interpret incorrect proof structures.
So we use a compact closed category to interpret proof structures.
Fortunately, \( \ProfCat \) (as well as \( \ProfCat_{\omega} \)) is compact closed, so this approach works for our setting.

\tk{to do: give an interpretation of MLL proof structure.}
\tk{Show the coherence by giving a direct interpretation (= \emph{experiments}) and comparing it with the standard interpretation}


A \emph{creed assignment} \( \varrho \) is a mapping from propositional variables to groupoids with creeds.
We write \( |\varrho| \) for the induced \emph{groupoid assignment}, i.e.~a mapping from positional variables to groupoinds.

The interpretation of a proof structure is defined by induction on the number of nodes.
\tk{to do}

The definition of this semantics depends not on the proof structures themselves but on the way they are constructed; however, it is in fact well-defined with respect to the proof structures. This follows from the coherence of compact closed categories. Here, partly for convenience in later discussions, we demonstrate this by providing a direct interpretation from the proof structures. This direct interpretation corresponds to the concept of Girard's experiment.


\subsection{Creed and Kit as Correctness}
We characterise the correctness of a proof structer \( \pi \vdash \Gamma \) in terms of creed/kit.
The proof structure \( \pi \) induces a profunctor \( \sem{\pi}_\varrho \colon I \profarrow \sem{\Gamma}_\varrho \), given an interpretation \( \varrho \) of propositional variables.
We associate a Boolean creed to \( \varrho(\PropVar) \) for each propositional variable \( \PropVar \), resulting in an assignment \( \hat{\varrho} \) of propositional variables in \( \SProf \).
We check whether \( \sem{\pi}_\varrho \) is stable, i.e.~whether \( \sem{\pi}_\varrho \in \SProf(I, \sem{\Gamma}_{\hat{\varrho}}) \).
The proof structure passes this test if the interpretation is stable for every choice of \( \varrho \) and \( \hat{\varrho} \).

\begin{definition}[Creed-Kit criterion]
    A cut-free proof structure \( \pi \) satsifies the \emph{creed-kit criterion} if
    \begin{quote}
        \( \sem{\pi}_{\varrho} \models \sem{\Gamma}_{\hat{\varrho}} \) for every assignment \( \hat{\varrho} \) of propositional variables in \( \SProf \).
        \tk{to do: introduce the notation}
    \end{quote}
    Here \( \varrho \) is an assiment of propositional variables in \( \ProfCat \) induced by \( \hat{\varrho} \).
    %
    \qed
\end{definition}

\begin{theorem}\label{thm:creed-kit-correctness-without-cut}
    A cut-free proof structure is correct with respect to \( \mathsf{MLL+Mix} \) if and only if it satsifies the creed-kit criterion.
\end{theorem}
\begin{proof}
    The left-to-right direction is trivial since \( \SProf \) is a model of \( \mathsf{MLL+Mix} \).
    We prove the converse.

    Assume an incorrect proof structure \( \pi \).
    By \cref{thm:weak-danos-regnier}, there exists a Danos-Regnier switching of \( \pi \) such that the associated graph has a cycle.
    Fix a Danos-Regnier switching with a cycle in the associated groph and choose a node \( N \) such that (1) it is involved in a cycle and that (2) it is a root or its parent is not involved in any cycle.\tk{add a figure}
    We can assume without loss of generality that \( N \) is reachable from a root in the associated graph; if it is not reachable, change the switching in the unique path from \( N \) to its root in the syntax tree.\tk{add a figure}
    We search the associated graph starting from \( N \) in the depth-first mannar by the following rules:
    \begin{enumerate}
        \item 
        Each edge is traversed at most twice (at once in each direction).
        \item
        If the destination of an edge being traversed is a node that has already been visited, we immediately go back the edge (without further explorating the destination at this moment).
        \item
        If the current node is a \( \otimes \)-node and there are multiple unexplored edges from the node, the next edge is chosen based on the following priority: the edge leading to the left child, the edge leading to the right child, and finally, the edge leading to the parent.
    \end{enumerate}
    For each traverssed edge, we record the direction in which it was first traversed.
    \tk{add a figure}

    \begin{claim*}
        The start node \( N \) must be annotated as follows: \tk{add a figure}
    \end{claim*}
    \begin{proof}
        \tk{to do: here}
    \end{proof}
\end{proof}



\subsection{Matrix-Based Correctness}
As we have seen in \cref{sec:matrix}, creeds and kits are closely related to the functoriality of the matrix representation.
So the correctness criterion (\cref{thm:creed-kit-correctness-without-cut}) based on creeds/kits suggests a correctenss criterion based on the well-behavedness of the matrix representation.

\begin{theorem}
    An \(\mathsf{MLL}\) proof structure \( \pi \vdash A_1, \dots, A_n \) is a correct proof-net for \( \mathsf{MLL+MIX} \) if and only if, for every substitution \( \vartheta \) of propositional variables in \( \pi \) and every correct proof-net \( \xi \vdash (A_i\vartheta)^{\bot} \), the proof-structure \( \mathsf{cut}(\pi\vartheta, \xi) \vdash (A_1, \dots, A_{i-1}, A_{i+1}, \dots, A_n)\vartheta \) is matrixable\tk{to do: give a terminology}. 
\end{theorem}

\begin{corollary}
    \( \MatTrans{\sem{\pi}} = \rsem{\pi} \) for every sequent calculus proof \( \pi \) of LL + MIX.
\end{corollary}
\begin{corollary}
    \( \MatTrans{\sem{\pi}} = \rsem{\pi} \) for a correct MLL + MIX proof \( \pi \). 
\end{corollary}

