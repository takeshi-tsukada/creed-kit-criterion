\section{Correctness for \textsf{MLL+MIX} without Cut}

\tk{to do: give an interpretation of MLL proof structure.  Show the coherence by giving a direct interpretation (= \emph{experiments}) and comparing it with the standard interpretation}

An \emph{assignment} \( \varrho \) is a mapping from propositional variables to groupoids.

\begin{theorem}
    Let \( \pi \) be a cut-free proof structure for \textsf{MLL+MIX} (without unit).
    It is correct if and only if the interpretation \( \sem{\pi}_\varrho \) in \( \mathbf{Prof} \) satisfies the creed-kit criterion for every \( \varrho \).
    % An \textsf{MLL} proof structure (without unit) satisfies the creed-kit criterion if and only if it is a correct proof-net for \textsf{MLL+MIX}.
\end{theorem}


\begin{theorem}
    An MLL proof structure (without unit) \( \pi \) is a correct proofnet for MLL + MIX if and only if, for every substitution \( \vartheta \) of propositional variables in \( \pi \) and every correct proof-net \( \pi' \), \( \rsem{\mathsf{cut}(\pi\vartheta, \pi')} = \MatTrans{\sem{\mathsf{cut}(\pi\vartheta, \pi')}} \). 
\end{theorem}

\tk{to do: creed-kit criterion for non-cut-free proof structures}

\begin{corollary}
    \( \MatTrans{\sem{\pi}} = \rsem{\pi} \) for every sequent calculus proof \( \pi \) of LL + MIX.
\end{corollary}
\begin{corollary}
    \( \MatTrans{\sem{\pi}} = \rsem{\pi} \) for a correct MLL + MIX proof \( \pi \). 
\end{corollary}



\section{Correctness for MLL+MIX with \textsf{cut}}


\newcommand{\ProfCat}{\mathbf{Prof}}
\newcommand{\CohProf}{\mathbf{CohProf}}
\newcommand{\TotProf}{\mathbf{TotProf}}
\begin{definition}
    The \emph{partial orthogonality} \( \orthogonal^{\mathit{coh}} \) and \emph{total orthogonality} \( \orthogonal^{\mathit{tot}} \) on \( \ProfCat \) are defined as follows:
    for \( F \in \ProfCat(I, \Grp) \) and \( G \in \ProfCat(\Grp, I) \),
    \begin{align*}
        F \orthogonal^{\mathit{coh}} G &\quad\defiff\quad F \orthogonal G \mbox{ and } G \circ F \le \ident_I \\ 
        F \orthogonal^{\mathit{coh}} G &\quad\defiff\quad F \orthogonal G \mbox{ and } G \circ F \cong \ident_I,
    \end{align*}
    where for \( H \in \ProfCat(I,I) = (I^\op \times I \to \Set) \cong \Set \), we write \( H \le \ident_I \) if \( H \) is empty or singleton.
    %
    \qed
\end{definition}

We write \( \CohProf \) and \( \TotProf \) for the tight orthogonality categories \( T_{\orthogonal^{\mathit{coh}}}(\ProfCat) \) and \( T_{\orthogonal^{\mathit{tot}}}(\ProfCat) \).
% be the slack orthogonality category, where \( f \orthogonal g \) if and only if \( g \circ f = \ident_I \) and this pair satisfies the jointly freeness condition.

\newcommand{\Coh}{\mathbf{Coh}}
\newcommand{\Tot}{\mathbf{Tot}}
\begin{proposition}
    There exists functors \( \CohProf \longrightarrow \Coh \) and \( \TotProf \longrightarrow \Tot \) that preserve the LL structures.
    %
    \qed
\end{proposition}

\begin{theorem}
    Let \( \pi \) be a proof structure for \textsf{MLL+MIX+CUT}.
    It is correct if and only if the interpretation \( \sem{\pi}_{|\varrho|} \) in \( \ProfCat \) is coherent for every assignemnt \( \varrho \) of groupoids with coherence.
    % An \textsf{MLL} proof structure (without unit) satisfies the creed-kit criterion if and only if it is a correct proof-net for \textsf{MLL+MIX}.
    A similar statement holds for the totality.
\end{theorem}
\begin{proof}
    The left-to-right direction is obvious since the tight orthogonality category is \( * \)-autonomous~\cite{Hyland2003}.\tk{to do: details, e.g., page and lem/prop numbers}
\end{proof}
