\section{Well-Behavedness of Matrix Representation}

\newcommand{\MatTrans}[1]{\mathcal{M}({#1})}
\newcommand{\rsem}[1]{(\!|{#1}|\!)}
\newcommand{\SProf}{\mathbf{SProf}}
\newcommand{\WRel}{\mathbf{WRel}}
\newcommand{\Real}{\mathbb{R}}


\tk{to do: explain the background, and the ``visibility'' condition in \cite{Tsukada2018}}

\tk{to do: define \( \SProf_\omega \)}

\tk{to do: define the matrix representation}

\newcommand{\TerminalCategory}{\mathbf{1}}
\begin{example}
    Let \( G \) be a finite group, \( H \) be its subgroup, \( \Grp \) be a single-object groupoid with \( \Grp(a, a) = G \), and \( \eta \colon \TerminalCategory \profarrow \Grp^{\op} \times \Grp \) and \( \epsilon \colon \Grp^{\op} \times \Grp \profarrow \TerminalCategory \) be profunctors defined by \( \varphi(\star, ({-}, {+})) = \Grp({-}, {+}) \) and \( \psi(({+}, {-}), \star) = \Grp({-}, {+}) \).
    Then \( \MatTrans{\eta}_{\star, (a,a)} = \#\Grp(a,a) = \#G \) and \( \MatTrans{\epsilon}_{(a,a), \star} = (1/\#\Grp(a,a))\#\Grp(a,a) = 1 \), so \( (\MatTrans{\epsilon} \circ \MatTrans{\eta})_{\star, \star} = \#G \).
    The profunctor composition is given by \( (\epsilon \circ \eta)(\star, \star) = \big( \epsilon((a,a), \star) \times \eta(\star, (a,a)) \big) / {\sim} \), where \( (y, x) \sim (y', x') \) if and only if there exist \( g, h \in G \) such that \( y' = y \cdot (g,h) = g y h \) and \( x' = (g^{-1}, h^{-1}) \cdot x = h^{-1} x g^{-1} \).
    Then \( (y, x) \sim (1_G, yx) \) by taking \( g = 1_G \) and \( h = y^{-1} \), so we can choose representatives from those of the form \( (1_G, x) \).
    Furthermore, \( (1_G, x) \sim (1_G, x') \) if and only if there exists \( f \in G \) such that \( f^{-1} x f = x' \), i.e.~\( x \) and \( x' \) belong to the same conjugacy class.
    Hence \( (\epsilon \circ \eta)_{\star, \star} \) consists of the set of all conjugacy classes of \( G \).
    So \( \MatTrans{\epsilon} \circ \MatTrans{\eta} \neq \MatTrans{\epsilon \circ \eta} \), provided that \( G \) is not abelian.
    %
    \qed
\end{example}

\tk{confer: Joyal, definition 7, Intro-Species text}

\begin{theorem}
    % Let \( \creed \), \( \creedb \) and \( \creedc \) be groupoids with creeds.
    Let \( F \colon \creed \profarrow \creedb \) and \( G \colon \creedb \profarrow \creedc \) be stable profunctors.
    Then \( \MatTrans{G} \circ \MatTrans{F} = \MatTrans{G \circ F} \).
\end{theorem}
\begin{proof}
    Let \( a \in \Grp \), \( b \in \Grpb \) and \( c \in \Grpc \) be representatives of isomorphism classes.
    By definition, \( \MatTrans{F}_{a,b} = (1/\#\Grp(a,a)) \# F(a,b) \) and \( \MatTrans{G}_{b,c} = (1/\#\Grpb(b,b)) \#G(b,c) \).
    So \( (\MatTrans{G} \circ \MatTrans{F})_{a,c} = (1/\#\Grp(a,a)) \sum_{b} (1/\#\Grpb(b,b)) \# G(b,c) \cdot \# F(a,b) \) (where \( b \) ranges over representatives of isomorphism classes of objects in \( \Grpb \)).
    By definition, \( (G \circ F)(a,c) = \int^{b \in \Grpb} G(b,c) \times F(a,b) = (\coprod_{b \in \Grpb} G(b,c) \times F(a,b)) / {\sim} \), where \( (y,x) \sim (y', x') \Longleftrightarrow \exists f \colon b' \to b.\ (y', x') = (y \cdot f, f^{-1} \cdot x) \).
    For each representative \( b \) and \( (y,x) \in G(b,c) \times F(a,b) \), the equvalence class \( \{ (y', x') \in G(b,c) \times F(a,b) \mid (x,y) \sim (x', y') \} \) consists of \( \# \Grpb(b,b) \) elements because the action \( (x,y) \mapsto (y \cdot f, f^{-1} \cdot x) \) is free by the stabilities of \( F \) and \( G \).
    So the required equation holds.
\end{proof}
Note that the freeness of the action \( (y,x) \mapsto (y \cdot f, f^{-1} \cdot x) \) has been observed in \cite{Fiore2024}\tk{todo: check}

\begin{corollary}
    \( \MatTrans{{-}} \) is a functor from \( \SProf_\omega \) to \( \WRel_{\Real_+^\infty} \).
\end{corollary}

\begin{theorem}
    Let \( F \colon A \profarrow B \) and \( G \colon B \profarrow C \).
    If \( \MatTrans{G} \circ \MatTrans{F} = \MatTrans{G \circ F} \), then \( \creed^{F(a, {-})} \orthogonal \creed^{G({-}, c)} \) for every \( a \in A \) and \( c \in C \).
\end{theorem}
\begin{proof}
    Recall that \( F \cong \coprod_i \InducedProf{G_i} \) and \( G \cong \coprod_j \InducedProf{H_j} \).
    The \( (G \circ F) \cong \coprod_{i,j} \InducedProf{H_j} \circ \InducedProf{G_i} \).
    It is not diffucult to see that \( \MatTrans{\InducedProf{H_j}} \circ \MatTrans{\InducedProf{G_i}} \le \MatTrans{\InducedProf{H_j} \circ \InducedProf{G_i}} \) (here the order is component-wise).
    We prove that \( \MatTrans{\InducedProf{H_j}} \circ \MatTrans{\InducedProf{G_i}} < \MatTrans{\InducedProf{H_j} \circ \InducedProf{G_i}} \) if \( G_i \) and \( H_j \) are not orthogonal.\tk{to do: explain the meaning}

    \tk{to do}
\end{proof}

\begin{theorem}
    Let \( F \colon A \profarrow B \) and \( G \colon B \profarrow C \).
    Then \( \MatTrans{G} \circ \MatTrans{F} = \MatTrans{G \circ F} \) if and only if \( \creed^{F(a, {-})} \orthogonal \creed^{G({-}, c)} \) for every \( a \in A \) and \( c \in C \).
\end{theorem}





