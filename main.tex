\documentclass[conference]{IEEEtran}
\IEEEoverridecommandlockouts
% The preceding line is only needed to identify funding in the first footnote. If that is unneeded, please comment it out.
%Template version as of 6/27/2024

\newif\ifdraft\drafttrue
% \newif\ifdraft\draftfalse

\usepackage{cite}
\usepackage{amsmath,amssymb,amsfonts}
\usepackage{mathtools}
\usepackage{stmaryrd}
\usepackage{algorithmic}
\usepackage{graphicx}
\usepackage{textcomp}
\usepackage{cleveref}
\usepackage{xcolor}
\def\BibTeX{{\rm B\kern-.05em{\sc i\kern-.025em b}\kern-.08em
    T\kern-.1667em\lower.7ex\hbox{E}\kern-.125emX}}

% for LL symbols
\usepackage{cmll}

\usepackage{amsthm}
\theoremstyle{plain}
\newtheorem{theorem}{Theorem}
\newtheorem{lemma}[theorem]{Lemma}
\newtheorem{corollary}[theorem]{Corollary}
\newtheorem{proposition}[theorem]{Proposition}
\newtheorem{conjecture}[theorem]{Conjecture}
\newtheorem*{proposition*}{Proposition}
\newtheorem*{claim*}{Claim}

\theoremstyle{definition}
\newtheorem{definition}[theorem]{Definition}
\newtheorem{example}[theorem]{Example}
\newtheorem*{problem*}{Problem}
\newtheorem*{question*}{Question}
\newtheorem{question}{Question}

\theoremstyle{remark}
\newtheorem{remark}[theorem]{Remark}
\newtheorem*{notation}{Notation}


\definecolor{DarkGreen}{RGB}{0,120,0}
\definecolor{DarkBlue}{RGB}{0,0,200}
\definecolor{Cyan}{RGB}{0,139,139}



% Luke's macros
\newcommand\set[1]{\{#1\}}
\newcommand\simcl[1]{\widetilde{#1}}
\newcommand\Deri{\mathcal{D}}
\newcommand\Var{\mathit{Var}}
\newcommand\lohide[1]{}
\newcommand\FCF{\hbox{\rm FCF}}
%\newcommand\ndlamy{{\lambda\Y_\oplus}}
\newcommand\ndlamy{{\lambda_{\rm nd}\Y}}

\newcommand{\Powerset}{\mathcal{P}}

\newcommand{\Int}{\mathbb{Z}}
\newcommand{\defeq}{:=}
\newcommand{\defiff}{:\Longleftrightarrow}
\newcommand{\orthogonal}{\mathrel{\bot}}

% \newcommand{\llpar}{\mathbin{\rotatebox[origin=c]{180}{\&}}}
\newcommand{\llpar}{\invamp}

\newcommand{\profarrow}{\mathrel{\relbar\joinrel\mapstochar\rightarrow}}

% general
\newcommand{\red}{\longrightarrow}
\newcommand{\evalto}{\hookrightarrow}
\newcommand{\ident}{\mathrm{id}}
%\newcommand{\sem}[1]{[\![{#1}]\!]}
\newcommand{\sem}[1]{\llbracket{#1}\rrbracket}
\newcommand{\op}{\mathrm{op}}

\newcommand{\TmRef}{\lhd}
\newcommand{\TyRef}{\lhd}
\newcommand{\TmRefInv}{\rhd}

\newcommand{\udstackrel}[3]{\mathrel{\mathop{#3}\limits^{#1}_{#2}}}

\newcommand{\Sym}[1]{\mathfrak{S}_{#1}}

\newcommand{\Seq}[1]{\langle{#1}\rangle}

\newcommand{\Boehm}[1]{\mathit{BT}({#1})}

% category
\newcommand{\FreeTensor}{\mathbb{P}}
\newcommand{\Set}{\mathbf{Set}}

\newcommand{\Rel}{\mathbf{Rel}}
\newcommand{\MRel}{\mathbf{Rel}_{!}}

\newcommand{\ESP}{\mathbf{ESP}}
\newcommand{\Prof}{\mathbf{Prof}}


% terms
\newcommand{\BTT}{\mathtt{t}\!\mathtt{t}}
\newcommand{\BFF}{\mathtt{f}\!\mathtt{f}}

\newcommand{\rembranch}[1]{|{#1}|}

% Types
\newcommand{\TBool}{\mathtt{bool}}
\newcommand{\TNat}{\mathtt{nat}}
\newcommand{\qubit}{\mathbf{qubit}}
\newcommand{\TB}{\mathtt{o}}
\newcommand{\TBB}{\star}

% Terms
\newcommand{\TSkip}{\mathtt{skip}}
\newcommand{\TLet}[3]{\mathtt{let}\,{#1} = {#2}\,\mathtt{in}\,{#3}}
\newcommand{\TNondet}[2]{{#1} \oplus {#2}}
\newcommand{\TBL}[1]{{#1}\oplus\bullet}
\newcommand{\TBR}[1]{\bullet\oplus{#1}}
%% \newcommand{\TBL}[1]{\mathtt{L}; {#1}}
%% \newcommand{\TBR}[1]{\mathtt{R}; {#1}}
\newcommand{\TMatch}[3]{\mathtt{match}\;{#1}\;\mathtt{with}\;{#2} \mapsto {#3}}
%\newcommand{\TIf}[3]{\mathtt{if}\;{#1}\;\mathtt{then}\;{#2}\;\mathtt{else}\;{#3}}
\newcommand{\TIf}[3]{\mathtt{if}\;{#1}\;{#2}\;{#3}}
%% \newcommand{\TIfThen}[2]{\mathtt{if}\;{#1}\;\mathtt{then}\;{#2}}
%% \newcommand{\TIfElse}[2]{\mathtt{if}\;{#1}\;\mathtt{else}\;{#2}}
\newcommand{\TIfThen}[2]{\TIf{#1}{#2}{\bullet}}
\newcommand{\TIfElse}[2]{\TIf{#1}{\bullet}{#2}}
\newcommand{\TSplit}{\mathtt{split}}
\newcommand{\TLetrec}[3]{\mathtt{letrec}\,{#1} = {#2}\,\mathtt{in}\,{#3}}
\newcommand{\TMeas}{\mathtt{meas}}
\newcommand{\TNew}{\mathtt{new}}
\newcommand{\TIszero}{\mathtt{iszero}}
\newcommand{\TSucc}{\mathtt{succ}}
\newcommand{\TPred}{\mathtt{pred}}

\newcommand{\CNat}[1]{\underline{#1}}
\newcommand{\CSucc}{\mathtt{s}}
\newcommand{\CPred}{\mathtt{p}}
\newcommand{\Ifzero}[3]{\mathtt{ifzero}\,{#1}\,\mathtt{then}\,{#2}\,\mathtt{else}\,{#3}}

\newcommand{\Y}{\mathbf{Y}}
\newcommand{\CRand}{\mathtt{rand}}

\newcommand{\NF}{\mathsf{nf}}

\newcommand{\rcase}[3]{\mathtt{case}\,{#1}\,\mathtt{of}\,{#2}\,\Rightarrow\,{#3}}


%% Resource calculus
\newcommand{\flex}[1]{(\!|{#1}|\!)} % the translation from rigid terms to flexible terms
\newcommand{\Taylor}[1]{{#1}^*}

\newcommand{\Unmark}[1]{\ulcorner{#1}\urcorner}


\newcommand{\Deriv}{\mathcal{D}}
\newcommand{\CDev}[1]{\mathcal{C}({#1})}

\newcommand{\isoact}[2]{{#2}[{#1}]}



% the marker to indicate what is changed
\ifdraft
\newcommand{\tkchanged}[1]{{\color{red}{#1}}}
\newcommand{\tk}[1]{{\textcolor{DarkGreen}{[{#1}---Tsukada]}}}
\else
\newcommand{\tkchanged}[1]{{#1}}
\newcommand{\tk}[1]{}
\fi


\begin{document}


% \title{On Creed-Kit Criterion on Symmetries \\ in Profunctor Model of Linear Logic
% %\thanks{Identify applicable funding agency here. If none, delete this.}
% }

\title{On the Creed-Kit Criterion \\ in the Profunctor Model of Linear Logic}

% \author{\IEEEauthorblockN{1\textsuperscript{st} Given Name Surname}
% \IEEEauthorblockA{\textit{dept. name of organization (of Aff.)} \\
% \textit{name of organization (of Aff.)}\\
% City, Country \\
% email address or ORCID}
% \and
% \IEEEauthorblockN{2\textsuperscript{nd} Given Name Surname}
% \IEEEauthorblockA{\textit{dept. name of organization (of Aff.)} \\
% \textit{name of organization (of Aff.)}\\
% City, Country \\
% email address or ORCID}
% \and
% \IEEEauthorblockN{3\textsuperscript{rd} Given Name Surname}
% \IEEEauthorblockA{\textit{dept. name of organization (of Aff.)} \\
% \textit{name of organization (of Aff.)}\\
% City, Country \\
% email address or ORCID}
% \and
% \IEEEauthorblockN{4\textsuperscript{th} Given Name Surname}
% \IEEEauthorblockA{\textit{dept. name of organization (of Aff.)} \\
% \textit{name of organization (of Aff.)}\\
% City, Country \\
% email address or ORCID}
% \and
% \IEEEauthorblockN{5\textsuperscript{th} Given Name Surname}
% \IEEEauthorblockA{\textit{dept. name of organization (of Aff.)} \\
% \textit{name of organization (of Aff.)}\\
% City, Country \\
% email address or ORCID}
% \and
% \IEEEauthorblockN{6\textsuperscript{th} Given Name Surname}
% \IEEEauthorblockA{\textit{dept. name of organization (of Aff.)} \\
% \textit{name of organization (of Aff.)}\\
% City, Country \\
% email address or ORCID}
% }

\maketitle

\begin{abstract}
    The bicategory of profunctors can be seen as a categorification of the relational model of linear logic.
    This paper studies the action of symmetries in the profunctor interpretation of correct and incorrect nets of linear logic.
\end{abstract}

\begin{IEEEkeywords}
    profunctor, creed/kit, correctness criterion, Taylor expansion, generating series
\end{IEEEkeywords}

% \newcommand{\ident}{\mathrm{id}}
% \newcommand{\creed}{\coh}
\newcommand{\creed}{\mathcal{A}}
\newcommand{\creedb}{\mathcal{B}}
\newcommand{\creedc}{\mathcal{C}}
\newcommand{\Creed}{\mathit{Creed}}
\newcommand{\Grp}{\mathbb{A}}
\newcommand{\Grpb}{\mathbb{B}}
\newcommand{\Grpc}{\mathbb{C}}
% \newcommand{\op}{\mathit{op}}

\newcommand{\kit}{\mathcal{A}}
\newcommand{\kitb}{\mathcal{B}}
\newcommand{\kitc}{\mathcal{C}}

\newcommand{\InducedProf}[1]{\hat{#1}}
\newcommand{\Subgroup}{\mathit{Subgroup}}


\section{Preliminaries}

\subsection{Presheaves and Profunctors}

\newcommand{\Object}{\mathit{obj}}
\newcommand{\Skeleton}{\mathit{sk}}

A \emph{groupoid} \( \Grp \) is a category in which every morphism is invertible.
In particular, the set \( \Grp(a,a) \) of endo-morphisms is a group for every object \( a \in \Grp \).
A groupoid can be seen as a set in which each element is equipped with a group: a groupoid \( \Grp \) is equivalent to its skeleton \( \Skeleton(\Grp) \), which is a set \( \Object(\Skeleton(\Grp)) \) equipped with a group \( \Skeleton(\Grp)(a, a) \) for each \( a \in \Object(\Skeleton(\Grp)) \).

A \emph{(contravariant) presheaf} on a groupoid \( \Grp \) is a functor \( \Grp^{\op} \longrightarrow \Set \).
\tk{to do: the notation of the action}

For groupoids \( \Grp \) and \( \Grpb \), a \emph{profunctor} \( F \colon \Grp \profarrow \Grpb \) is a functor \( F \colon \Grp^{\op} \times \Grpb \longrightarrow \Set \).
The definition works for general categories \( \Grp \) and \( \Grpb \), but this paper focus on profunctors between groupoids.
\tk{to do: the notation of the action}

\tk{to do: describe the category of profunctors (note: identification of profunctors by iso.).  Describe its compact closed structure and \( ! \)}

\subsection{Presheaves and Profunctors as Sums of Subgroups}
Presheaves and profunctors on groupoids have an useful decomposition.
Let \( \Grp \) be a groupoid, \( a \in \Grp \), and \( G \) be a subgroup of \( \Grp(a,a) \).
We write \( \InducedProf{G} \) for the presheaf defined by:
\begin{align*}
    \InducedProf{G}(a')
    &\quad\defeq\quad
    \{ G \cdot f \mid f \in \Grp(a', a) \}
    \\
    \InducedProf{G}(f)
    &\quad\defeq\quad
    X \mapsto (X \cdot f)
\end{align*}
where for \( X \subseteq \Grp(a', a) \) and \( f \in \Grp(a'', a') \), \( G \cdot f \) is the subset of \( \Grp(a'', a) \) defined by \( (X \cdot f) \defeq \{ x \circ f \mid x \in X \} \).

\begin{proposition}
    The stabiliser groups of \( \InducedProf{G} \) are conjugacies of \( G \).
    %
    \qed
\end{proposition}

\newcommand{\Stabiliser}{\mathsf{fix}}

Let \( F \) be a presheaf on \( \Grp \).
We introduce an equivalence relation \( \sim \) on \( \coprod_a F(a) \defeq \{ (a, x) \mid x \in F(a) \} \) by
\begin{equation*}
    (a, x) \sim (a', x')
    \quad\defiff\quad
    \exists f \in \Grp(a', a).\: x \cdot f = x'.
\end{equation*}
Let \( \mathcal{R} \) be a set of representatives of the equivalence classes of \( \sim \).
For \( (a, x) \in \mathcal{R} \), let \( \Stabiliser(x) \in \Subgroup(\Grp(a,a)) \) be the stabilisers of \( x \), i.e.~\( \Stabiliser(x) \defeq \{ f \in \Grp(a,a) \mid x \cdot f = x \} \).

\begin{proposition}
    \( F \cong \coprod_{(a,x) \in \mathcal{R}} \InducedProf{\Stabiliser(x)} \).
\end{proposition}
\begin{proof}
    We give natural transformations \( \eta \colon F \Rightarrow \coprod_{(a,x) \in X} \InducedProf{G_x} \) and \( \mu \colon \coprod_{(a,x) \in X} \InducedProf{G_x} \Rightarrow F \) as follows.

    Given \( y \in F(b) \), there exist \( (a,x) \in X \) and \( f \in \Grp(b,a) \) such that \( y = x \cdot f \).
    Then we define \( \mu_b(y) \defeq G_x \cdot f \in \InducedProf{G_x} \subseteq \coprod_{(a,x) \in X} \InducedProf{G_x} \).
    This operation is well-defined: the choice of \( (a,x) \in X \) is unique and, if \( g \in \Grp(b,a) \) also satisfies \( y = x \cdot f \), then \( x = x \cdot f \cdot g^{-1} \), so \( f \circ g^{-1} \in G_x \) and \( G_x \cdot f = G_x \cdot g \).
    The naturality is obvious.

    Given \( (a,x) \in X \) and \( X \in \InducedProf{G_x}(b) \), we have \( X = G_x \cdot f \) for some \( f \in \Grp(b,a) \).
    Then \( \mu_b(X) \defeq x \cdot f \).
    The well-definedness and naturality is trivial.

    It is easy to see that \( \mu \circ \eta = \ident \) and \( \eta \circ \mu = \ident \).
\end{proof}
\begin{corollary}
    Every presheaf \( F \in \mathit{psh}(\Grp) \) is isomorphic to a coproduct \( \coprod_{i \in I} \InducedProf{G_i} \) of profunctors induced by subgroups \( G_i \subseteq \Grp(a_i, a_i) \).
\end{corollary}

A similar argument applies to profunctors.
A subgroup \( G \) of \( \Grp^\op(a,a) \times \Grpb(b,b) \) induces a profunctor \( \InducedProf{G} \colon \Grp \profarrow \Grpb \) by 
\begin{align*}
    \InducedProf{G}(a', b')
    &\quad\defeq\quad
    \{ g \cdot G \cdot f \mid f \in \Grp(a', a), g \in \Grpb(b, b') \}
    \\
    \InducedProf{G}(f, g)
    &\quad\defeq\quad
    X \mapsto (g \cdot X \cdot f),
\end{align*}
where \( (g \cdot X \cdot f) \defeq \{ g \circ x \circ f \mid x \in X \} \).
A profunctor \( F \colon \Grp \profarrow \Grpb \) induces an equivalence relation \( \sim \) on \( \coprod_{a,b} F(a,b) \defeq \{ (a, x, b) \mid x \in F(a, b) \} \) given by
\( (a, x, b) \sim (a', x', b') \) if and only if \( g \cdot x \cdot f = x' \) for some \( f \in \Grp(a', a) \) and \( g \in \Grpb(b, b') \).
Let \( \mathcal{R} \) be a set of representatives of the equivalence classes of \( \sim \).
For \( (a, x) \in X \), let \( \Stabiliser(x) \in \Subgroup(\Grp^\op(a,a) \times \Grpb(b,b)) \) be the stabilisers of \( x \), i.e.~\( \Stabiliser(x) \defeq \{ (f,g) \in \Grp^{\op}(a,a) \times \Grpb(b,b) \mid g \cdot x \cdot f = x \} \).

\begin{proposition}
    \( F \cong \coprod_{(a,x,b) \in \mathcal{R}} \InducedProf{\Stabiliser(x)} \).
    For \( F \colon \Grp \profarrow \Grpb \) and \( G \colon \Grpb \profarrow \Grpc \) with \( F \cong \coprod_i \InducedProf{F_i} \) and \( G \cong \coprod_j \InducedProf{G_j} \), we have \( G \circ F \cong \coprod_{i,j} (\InducedProf{G_j} \circ \InducedProf{F_i}) \).
    %
    \qed
\end{proposition}



\section{Creeds and Kits}
This section reviews the notions of \emph{creeds}~\cite{Taylor1989} and \emph{kits}~\cite{Fiore2024}.
\tk{Should we cite the coference verison (Fiore 2023?) as well?}
They describe ``allowed symmetries'' of presheaves and profunctors, i.e.~\( f \in \Grp(a,a) \) such that \( x \cdot f = x \).

\subsection{Creeds}
\begin{definition}[Creed]
    A \emph{creed} of a groupoid \( \Grp \) is a collection \( (\creed(a))_{a \in \Grp} \) of subsets \( \creed(a) \subseteq \Grp(a,a) \) of endo-morphisms such that
    \begin{itemize}
        \item \( \ident_a \in \creed(a) \), \tk{do we need this condition?}
        \item \( g \in \creed(a) \) implies \( g^k \in \creed(a) \) for every \( k \in \Int \), and
        \item \( g \in \creed(a) \) and \( f \in \Grp(a,b) \) implies \( f \circ g \circ f^{-1} \in \creed(b) \).
    \end{itemize}
    Note that \( \creed(a) \) is not necessarily closed under composition.
    We write \( \Creed(\Grp) \) for the set of all creeds of \( \Grp \).
    %
    \qed
\end{definition}

We regard the set \( \Creed(\Grp) \) of creeds as a poset by the component-wise inclusion: \( \creed \subseteq \creed' \) is defined as \( \forall a \in \Grp. \creed(a) \subseteq \creed'(a) \).
The set \( \Creed(\Grp) \) of creeds is closed under the union and intersection.
That means, if \( \creed_i \) is a collection of creeds parameterised by \( i \in I \), then the collections \( \creed = (\creed(a))_a \) and \( \creed' = (\creed'(a))_i \) given by \( \creed(a) \defeq \bigcup_i \creed_i(a) \) and \( \creed'(a) \defeq \bigcap_i \creed_i(a) \) are creeds.
We write \( \creed \defeq \bigcup_i \creed_i \) and \( \creed' \defeq \bigcap_i \creed_i \).
The minimum creed consists only of the identities, and the maximum creed has all endo-morphisms.

A presheaf \( P \in \mathit{psh}(\Grp) \) induces a creed \( \creed^P \) defined by
\begin{equation*}
    g \in \creed^P(a)
    \quad\mbox{if and only if}\quad
    \exists x \in P(a).\: x \cdot g = x.
\end{equation*}

\begin{proposition}
    Every creed is the induced creed of a presheaf.
\end{proposition}
\begin{proof}
    Let \( \Grp \) be a groupoid.
    For every endo-morphism \( g \in \Grp(a,a) \), we define a presheaf \( P_g \in \mathit{psh}(\Grp) \) by
    \begin{equation*}
        P_g(b)
        \:\defeq\:
        \big\{ \{ g^k \circ f \mid k \in \Int \} ~\big|~ f \in \Grp(b,a) \big\}
        \:\subseteq\:
        \Powerset(\Grp(b,a))
    \end{equation*}
    The action \( P_g(f) \) of \( f \colon b' \to b \) in \( \Grp \) is given by \( P_g(f)(X) := \{ h \circ f \mid h \in X \} \).
    The induced creed \( \creed^{P_g} \) contains \( g \) since \( \{ g^k \mid k \in \Int \} \in P_g(a) \) and \( P_g(g) \) fixes this element.
    We show that \( P_g \) is the minimum creed that contains \( g \).
    Assume \( f \in \creed^{P_g}(b) \).
    Then there exists \( h \in \Grp(b,a) \) such that \( \{ g^k \circ h \mid k \in \Int \} = \{ g^k \circ h \circ f \mid k \in \Int \} \).
    Then \( h \circ f = g^k \circ h \) for some \( k \in \Int \).
    Therefore \( f = h^{-1} \circ g^k \circ h \).
    Let \( \creed' \) be a creed such that \( g \in \creed'(a) \).
    Then \( g^k \in \creed'(a) \) and hence \( f = h^{-1} \circ g^k \circ h \in \creed'(a) \).
    Since \( f \in \creed^{P_g}(b) \) is arbitrary, this means that \( P_g \) is the minimum creed among those containing \( g \).

    Then a given \( \creed \) is the induced creed of the profunctor \( \coprod_{g \in \creed(a)} P_g \).
\end{proof}

\begin{definition}
    A presheaf \( P \) follows a creed \( \creed \) if and only if \( \creed^P(a) \subseteq \creed(a) \) for every \( a \in \Grp \).
    %
    \qed
\end{definition}

A pair \( (\creed, \creed') \in \Creed(\Grp) \times \Creed(\Grp^\op) \) of creeds are \emph{orthogonal}, written \( \creed \orthogonal \creed' \), if \( \creed(a) \cap \creed'(a) = \{ \ident_a \} \) for every \( a \in \Grp \) (where we identify a morphism \( f \colon a \longrightarrow b \) in \( \Grp \) with the corresponding morphism \( b \longrightarrow a \) in \( \Grp^\op \)).
The orthogonality is symmetric: \( \creed \orthogonal \creed' \) implies \( \creed' \orthogonal \creed \).

Given a creed \( \creed \) on \( \Grp \), we write \( \creed^\bot \) for the maximum creed on \( \Grp^\op \) orthogonal to \( \creed \).
That means,
\begin{equation*}
    \creed^\bot
    \quad\defeq\quad
    \bigcup \{ \creed' \in \Creed(\Grp^\op) \mid \creed \orthogonal \creed' \}.
\end{equation*}
Then \( ({-})^{\bot\bot} \) is a closure operator.
An explicit definition can be given by
\begin{equation*}
    g \in \creed^\bot(a)
    \quad\Leftrightarrow\quad
    \forall k \in \Int.\: g^k \notin (\creed(a) \setminus \{ \ident_a \})
\end{equation*}

\begin{example}
    Let \( \Grp \) be a single-object groupoid such that \( \Grp(\star, \star) = \Int_2 \times \Int_3 \) for the unique object \( \star \).
    Let \( \creed = \Int_2 \times \{0\} \) be a creed on \( \Grp \).
    Then \( \creed^\bot = \{0\}\times \Int_3 \).
    To see this, assume \( (n,m) \in \Int_2 \times \Int_3 \).
    We have \( (n,m) \in \creed^\bot \) if and only if, for every \( k \), if \( (kn, km) \neq (0,0) \), then \( (kn, km) \notin \creed \).
    It is easy to see that \( (0,m) \in \creed^\bot \).
    Assume that \( (n,m) \in \creed^\bot \) with \( n \neq 0 \).
    Then \( (3n, 3m) = (n, 0) \in \creed \) must hold, a contradiction.
    Hence \( \creed^\bot = \{ 0 \} \times \Int_3 \).
    %
    \qed
\end{example}

A creed \( \creed \) on \( \Grp \) divides endo-morphisms \( g \in \Grp(a,a) \) into four classes:
(1) the identity, which belongs to both \( \creed(a) \) and \( \creed^\bot(a) \),
(2) \emph{positive} ones \( g \in (\creed(a) \setminus \creed^\bot(a)) \),
(3) \emph{negative} ones \( g \in (\creed^\bot(a) \setminus \creed(a)) \), and
(4) \emph{neutral} ones \( g \in (\Grp(a,a) \setminus (\creed(a) \cup \creed^\bot(a))) \).
We write \( \creed^{\boxplus} \), \( \creed^{\boxminus} \) and \( \creed^\blacksquare \) for the set of positive, negative and neutral elements, respectively.
By using this notation, \( g \in \creed^\bot(a) \) if and only if \( \forall k \in \Int.\: g^k \notin \creed^\boxplus \).

Let \( \creed = (\Grp, \creed) \) and \( \creedb = (\Grpb, \creedb) \) be creeds.
We define creeds \( \creed \otimes \creedb \) and \( \creed \llpar \creedb \) on \( \Grp \times \Grpb \) by
\begin{align*}
    (\creed \otimes \creedb)(a,b) 
    &:= (\creed(a) \times \creedb(b))^{\bot\bot} \\
    (\creed \llpar \creedb)(a,b)
    &:= (\creed^\bot(a) \times \creedb^\bot(b))^\bot.
\end{align*}
% \begin{equation*}
%     (\creed \otimes \creedb)(a,b) := \{ ((f, g), (f', g')) \mid f \creed_\kit f', g \creed_\kitb g' \}^{\bot\bot}
% \end{equation*}
% \begin{equation*}
%     \creed \otimes \creedb := (\creed \times \creedb)^{\bot\bot}
% \end{equation*}
% \begin{equation*}
%     ({\creed_{\kit \llpar \kitb}}) := \{ ((f, g), (f', g')) \mid f \creed_\kit f', g \creed_\kitb g' \}^{\bot\bot}
% \end{equation*}
% \begin{equation*}
%     \creed \llpar \creedb := (\creed^\bot \times \creedb^\bot)^\bot
% \end{equation*}

\begin{remark}
    \tk{to do: give comparison between creed by Taylor and kit by Fiore+.  Our definition uses creeds, but the operators are similar to kit operators (and Taylor's operations are not directly follow Hyland-Schalk)}
    %
    \qed
\end{remark}

Hence \( (g, h) \in (\creed \llpar \creedb)(a,b) \) if and only if \( \{ (g^k, h^k) \mid k \in \Int \} \cap (\creed^\bot(a) \times \creedb^\bot(b)) = \{ (\ident_a, \ident_b) \} \) if and only if \( \forall k \in \Int. (g^k, h^k) \neq (\ident_a, \ident_b) \Longrightarrow (g^k \notin \creed^\bot(a) \vee h^k \notin \creedb^\bot(b)) \) if and only if \( \forall k \in \Int. (g^k, h^k) \neq (\ident_a, \ident_b) \Longrightarrow \exists m. (g^{km} \in \creed^\boxplus(a) \vee h^{km} \in \creedb^\boxplus(b)) \).

% \begin{align*}
%     & (g,h) \in (\creed \otimes \creedb)^{\boxplus}(a,b)
%     \\ &
%     \quad\Longleftrightarrow
%     \forall k \in \Int. \exists m \in \Int.\ g^{km} \in \creed^{\boxplus}(a) \vee h^{km} \in \creedb^{\boxplus}(b)
% \end{align*}

\begin{align*}
    & (g,h) \in (\creed \llpar \creedb)(a,b)
    \\ &
    \quad\Longleftrightarrow
    \forall k \in \Int. (g^k, h^k) = (\ident_a, \ident_b) \vee {}
    \\ &
    \quad\qquad\qquad \exists m \in \Int.\ g^{km} \in \creed^{\boxplus}(a) \vee h^{km} \in \creedb^
    {\boxplus}(b)
\end{align*}

Hence \( (g, h) \in (\creed \otimes \creedb)(a,b) \) if and only if \( \{ (g^k, h^k) \mid k \in \Int \} \cap (\creed(a) \times \creedb(b))^\bot = \{ (\ident_a, \ident_b) \} \) if and only if \( \forall k \in \Int. (g^k, h^k) \neq (\ident_a, \ident_b) \Longrightarrow (g^k, h^k) \notin (\creed(a) \times \creed(b))^\bot \) if and only if \( \forall k \in \Int. (g^k, h^k) \neq (\ident_a, \ident_b) \Longrightarrow \exists m. (g^{km} \in \creed^\boxplus(a) \wedge h^{km} \in \creedb^\boxplus(b)) \).

\begin{align*}
    & (g,h) \in (\creed \otimes \creedb)(a,b)
    \\ &
    \quad\Longleftrightarrow
    \forall k \in \Int. (g^k, h^k) = (\ident_a, \ident_b) \vee {}
    \\ &
    \quad\qquad\qquad \exists m \in \Int.\ g^{km} \in \creed^{\boxplus}(a) \wedge h^{km} \in \creedb^
    {\boxplus}(b)
\end{align*}

\begin{example}
    Let \( \Grp \) be a groupoid and \( \creed \) be a creed on \( \Grp \).
    For every \( h \in \Grp(a,a) \), we have \( (h,h) \in (\creed \llpar \creed^{\bot})(a,a) \).
    Let \( k \in \Int \) be an arbitrary integer and assume \( h^k \neq \ident_a \).
    If \( \{ h^{km} \mid m \in \Int \} \cap \creed(a)^\boxplus \neq \emptyset \), then \( (h,h) \in (\creed \llpar \creed^{\bot})(a,a) \).
    Otherwise, \( h^{k} \in \creed^\bot(a) \) and hence \( h^k \in (\creed^\bot)^\boxplus(a) \).
    So \( (h,h) \in (\creed \llpar \creed^{\bot})(a,a) \) in this case, too.
    %
    \qed
\end{example}

\tk{what is the problem when we define \( \creed \otimes \creedb := \creed \times \creedb \) and \( \creed \llpar \creedb := \{ (g,h) \mid g \in \creed \vee h \in \creedb \} \)?}

\subsection{Kits}
A kit is a correction of allowed symmetries.
It is similar to a creed, but more fine-grained.

\begin{definition}[Kit]
    Let \( \Grp \) be a groupoid.
    A \emph{kit} \( \kit \) is a collection \( \kit = (\kit(a))_{a \in \Grp} \) of sets of subgroups \( \kit(a) \subseteq \Subgroup(\Grp(a,a)) \) closed under conjugation: \( G \in \kit(a) \) and \( f \colon b \longrightarrow a \) in \( \Grp \), \( f^{-1} \cdot G \cdot f \in \kit(b) \).
    %
    \qed
\end{definition}

\begin{proposition}
    For a kit \( \kit \), the family \( \creed'(a) := \bigcup \kit(a) \) is a creed (provided that \( \kit(a) \neq \emptyset \) for every \( a \)).
\end{proposition}

\begin{proposition}
    For a creed \( \creed \), the family \( \kit'(a) := \{ G \in \Subgroup(\Grp(a,a)) \mid G \subseteq \creed(a) \} \) is a kit.
\end{proposition}

The translations in the above propositions form an idempotent Galois connection.
Here the idempotency means that the passege \( (\mathit{Creed}) \mapsto (\mathit{Kit}) \mapsto (\mathit{Creed}) \) is the identity.

\section{Well-Behavedness of Matrix Representation}

\newcommand{\MatTrans}[1]{\mathcal{M}({#1})}
\newcommand{\rsem}[1]{(\!|{#1}|\!)}
\newcommand{\SProf}{\mathbf{SProf}}
\newcommand{\WRel}{\mathbf{WRel}}
\newcommand{\Real}{\mathbb{R}}


\tk{to do: explain the background, and the ``visibility'' condition in \cite{Tsukada2018}}

\tk{to do: define \( \SProf_\omega \)}

\tk{to do: define the matrix representation}

\newcommand{\FinSet}{\mathbf{FinSet}}

Recall that a species (in the sense of Joyal~\cite{Joyal1983?}) is a functor \( \FinSet \longrightarrow \FinSet \).
The \emph{(exponential) generating series} \( \widehat{F} \) of a species \( F \colon \FinSet \longrightarrow \FinSet \) is a formal power series defined by
\begin{equation*}
    \widehat{F}(x)
    \quad\defeq\quad
    \sum_{n = 0}^{\infty} \dfrac{\# F(n)}{n!} x^n,
\end{equation*}
where \( n \) represents a set of \( n \) elements and \( \# X \) is the cardinality of \( X \).
The generating series \( \widehat{F} \) is easier to handle and has much information of the original species \( F \).
Many operations on species have the corresponding operations on generating series.

A species \( F \) induces a profunctor \( \FinSet \profarrow I \), which we write as \( F \).
Note that \( \FinSet \cong \mathop{!}I \).
The Klisli composition \( G \circ_! F \) of \( F, G \colon \mathop{!}I \profarrow I \) is called the \emph{substitution} of species~\cite{FioreSpecies}.
If \( F(0) = \emptyset \), the generating series of the substitution is the substitution of generating series~\cite[Lem???]{Joyal?}:
\begin{equation*}
    (\widehat{G \circ_! F})(x)
    \quad=\quad
    \widehat{G}(\widehat{F}(x)).
\end{equation*}
However, the condition \( F(0) = \emptyset \) is necessarily.
\begin{example}
    Let \( F \) be the species defined by \( F(0) \defeq \{ a, b \} \) and \( F(n) \defeq \emptyset \) for every \( n \ge 0 \).
    Let \( G \) be the species defined by \( G(2) \defeq 1 \) and \( G(n) \defeq \emptyset \) for every \( n \neq 2 \).
    Then \( \widehat{F}(x) = 2 \) and \( \widehat{G}(y) = y^2/2 \).
    So \( \widehat{G}(\widehat{F}(x)) = 2 \).
    The Kleisli lifting \( F^! \colon \mathop{!}I \profarrow \mathop{!}I \) of \( F \) is given by \( F(0,m) = \{ a, b \}^m \), \( F(0, \sigma)(x_1,\dots,x_m) = (x_{\sigma(1)}, \dots, x_{\sigma(m)}) \)\tk{inverse?} for a permutation \( \sigma \colon m \to m \) and \( (x_1,\dots,x_n) \in \{ a,b \}^m \), and \( F(n,m) = \emptyset \) for every \( n > 0 \).
    Then \( (G \circ_! F)(0) = \int^{m \in \mathop{!}I} G(m) \times F^!(0, m) = \{ a, b \}^2/{\sim} \) where \( (x_1,x_2) \sim (y_1,y_2) \) if \( (x_1,x_2) = (y_1,y_2) \) or \( (x_1,x_2) = (y_2,y_1) \).
    So \( (\widehat{G \circ_! F})(x) = 3 \).
    %
    \qed
\end{example}

\newcommand{\TerminalCategory}{\mathbf{1}}
\begin{example}
    Let \( G \) be a finite group, \( H \) be its subgroup, \( \Grp \) be a single-object groupoid with \( \Grp(a, a) = G \), and \( \eta \colon \TerminalCategory \profarrow \Grp^{\op} \times \Grp \) and \( \epsilon \colon \Grp^{\op} \times \Grp \profarrow \TerminalCategory \) be profunctors defined by \( \varphi(\star, ({-}, {+})) = \Grp({-}, {+}) \) and \( \psi(({+}, {-}), \star) = \Grp({-}, {+}) \).
    Then \( \MatTrans{\eta}_{\star, (a,a)} = \#\Grp(a,a) = \#G \) and \( \MatTrans{\epsilon}_{(a,a), \star} = (1/\#\Grp(a,a))\#\Grp(a,a) = 1 \), so \( (\MatTrans{\epsilon} \circ \MatTrans{\eta})_{\star, \star} = \#G \).
    The profunctor composition is given by \( (\epsilon \circ \eta)(\star, \star) = \big( \epsilon((a,a), \star) \times \eta(\star, (a,a)) \big) / {\sim} \), where \( (y, x) \sim (y', x') \) if and only if there exist \( g, h \in G \) such that \( y' = y \cdot (g,h) = g y h \) and \( x' = (g^{-1}, h^{-1}) \cdot x = h^{-1} x g^{-1} \).
    Then \( (y, x) \sim (1_G, yx) \) by taking \( g = 1_G \) and \( h = y^{-1} \), so we can choose representatives from those of the form \( (1_G, x) \).
    Furthermore, \( (1_G, x) \sim (1_G, x') \) if and only if there exists \( f \in G \) such that \( f^{-1} x f = x' \), i.e.~\( x \) and \( x' \) belong to the same conjugacy class.
    Hence \( (\epsilon \circ \eta)_{\star, \star} \) consists of the set of all conjugacy classes of \( G \).
    So \( \MatTrans{\epsilon} \circ \MatTrans{\eta} \neq \MatTrans{\epsilon \circ \eta} \), provided that \( G \) is not abelian.
    %
    \qed
\end{example}

\tk{confer: Joyal, definition 7, Intro-Species text}

\begin{theorem}
    % Let \( \creed \), \( \creedb \) and \( \creedc \) be groupoids with creeds.
    Let \( F \colon \creed \profarrow \creedb \) and \( G \colon \creedb \profarrow \creedc \) be stable profunctors.
    Then \( \MatTrans{G} \circ \MatTrans{F} = \MatTrans{G \circ F} \).
\end{theorem}
\begin{proof}
    Let \( a \in \Grp \), \( b \in \Grpb \) and \( c \in \Grpc \) be representatives of isomorphism classes.
    By definition, \( \MatTrans{F}_{a,b} = (1/\#\Grp(a,a)) \# F(a,b) \) and \( \MatTrans{G}_{b,c} = (1/\#\Grpb(b,b)) \#G(b,c) \).
    So \( (\MatTrans{G} \circ \MatTrans{F})_{a,c} = (1/\#\Grp(a,a)) \sum_{b} (1/\#\Grpb(b,b)) \# G(b,c) \cdot \# F(a,b) \) (where \( b \) ranges over representatives of isomorphism classes of objects in \( \Grpb \)).
    By definition, \( (G \circ F)(a,c) = \int^{b \in \Grpb} G(b,c) \times F(a,b) = (\coprod_{b \in \Grpb} G(b,c) \times F(a,b)) / {\sim} \), where \( (y,x) \sim (y', x') \Longleftrightarrow \exists f \colon b' \to b.\ (y', x') = (y \cdot f, f^{-1} \cdot x) \).
    For each representative \( b \) and \( (y,x) \in G(b,c) \times F(a,b) \), the equvalence class \( \{ (y', x') \in G(b,c) \times F(a,b) \mid (x,y) \sim (x', y') \} \) consists of \( \# \Grpb(b,b) \) elements because the action \( (x,y) \mapsto (y \cdot f, f^{-1} \cdot x) \) is free by the stabilities of \( F \) and \( G \).
    So the required equation holds.
\end{proof}
Note that the freeness of the action \( (y,x) \mapsto (y \cdot f, f^{-1} \cdot x) \) has been observed in \cite{Fiore2024}\tk{todo: check}

\begin{corollary}
    \( \MatTrans{{-}} \) is a functor from \( \SProf_\omega \) to \( \WRel_{\Real_+^\infty} \).
\end{corollary}

\begin{theorem}
    Let \( F \colon A \profarrow B \) and \( G \colon B \profarrow C \).
    If \( \MatTrans{G} \circ \MatTrans{F} = \MatTrans{G \circ F} \), then \( \creed^{F(a, {-})} \orthogonal \creed^{G({-}, c)} \) for every \( a \in A \) and \( c \in C \).
\end{theorem}
\begin{proof}
    Recall that \( F \cong \coprod_i \InducedProf{G_i} \) and \( G \cong \coprod_j \InducedProf{H_j} \).
    The \( (G \circ F) \cong \coprod_{i,j} \InducedProf{H_j} \circ \InducedProf{G_i} \).
    It is not diffucult to see that \( \MatTrans{\InducedProf{H_j}} \circ \MatTrans{\InducedProf{G_i}} \le \MatTrans{\InducedProf{H_j} \circ \InducedProf{G_i}} \) (here the order is component-wise).
    We prove that \( \MatTrans{\InducedProf{H_j}} \circ \MatTrans{\InducedProf{G_i}} < \MatTrans{\InducedProf{H_j} \circ \InducedProf{G_i}} \) if \( G_i \) and \( H_j \) are not orthogonal.\tk{to do: explain the meaning}

    \tk{to do}
\end{proof}

\begin{theorem}
    Let \( F \colon A \profarrow B \) and \( G \colon B \profarrow C \).
    Then \( \MatTrans{G} \circ \MatTrans{F} = \MatTrans{G \circ F} \) if and only if \( \creed^{F(a, {-})} \orthogonal \creed^{G({-}, c)} \) for every \( a \in A \) and \( c \in C \).
\end{theorem}






\section{Correctness of MLL+MIX}

\tk{to do: give an interpretation of MLL proof structure.  Show the coherence by giving a direct interpretation (= \emph{experiments}) and comparing it with the standard interpretation}

\begin{theorem}
    An MLL proof structure (without unit) satisfies the creed-kit criterion if and only if it is a correct proof-net for MLL + MIX.
\end{theorem}


\begin{theorem}
    An MLL proof structure (without unit) \( \pi \) is a correct proofnet for MLL + MIX if and only if, for every substitution \( \vartheta \) of propositional variables in \( \pi \) and every correct proof-net \( \pi' \), \( \rsem{\mathsf{cut}(\pi\vartheta, \pi')} = \MatTrans{\sem{\mathsf{cut}(\pi\vartheta, \pi')}} \). 
\end{theorem}

\tk{to do: creed-kit criterion for non-cut-free proof structures}

\section{Full completeness?}
Try to give a full completeness result.
It suffices to find a class of profunctors that can be obtained as an interpretation of proof-structure (not necessarily correct).
Then this condition together with the creed-kit criterion gives a characterisation of denotations of correct proof-structures.

\begin{theorem}
    \tk{Conjecture.  Even the statement is not concrete, as the notion of ``family'' has not yet formally given.}
    Every family of profunctors (that takes values in the slack orthogonality category) is definable.
\end{theorem}
\begin{proof}
    The slack orthogonality category can be mapped to the category of totality spaces, and a family is mapped to a family.
    By the definability of families of totality spaces~\cite{Loader1994}, there exists a bijection \( \varphi \) such that \( F[\vec{A}](a_1,\dots,a_n) \neq \emptyset \) if and only if \( a_i \cong a_{\varphi(i)} \) for every \( i \).
\end{proof}



\section{Towards a Criterion for MLL}

A \emph{positive good element} is inductively defined as follows:
\begin{itemize}
    \item For \( \creed \otimes \creedb \): a pair of positive good elements
    \item For \( \creed \llpar \creedb \): a pair of a positive good element and a negative good element
\end{itemize}
A \emph{negative good element} is inductively defined as follows:
\begin{itemize}
    \item For \( \creed \otimes \creedb \): a pair of a negative good element and a positive good element
    \item For \( \creed \llpar \creedb \): a pair of negative good elements
\end{itemize}

A profunctor is \emph{good} if it contains a good element in its creed.
(Equivalently, the stabiliser group of every element contains a good element.)

\begin{conjecture}
    An MLL proof structure (without unit) satisfies the strict creed-kit criterion if and only if it is a correct proof-net for MLL.
\end{conjecture}



\section{Free-Action Comonoid}
Given a category \( \mathcal{C} \), let \( {!}\mathcal{C} \) be the free symmetric strict unbiased monoidal category.
Its object is a finite sequence of objects of \( \mathcal{C} \) and its morphism is a permutation followed by a tuple of component-wise morphisms.
The functor \( u_{\mathcal{C}} \colon \mathcal{C} \longrightarrow !\mathcal{C} \) maps an object \( c \in \mathcal{C} \) to the singleton list \( \langle c \rangle \).
The functor \( m_{\mathcal{C}} \colon {!}{!}\mathcal{C} \longrightarrow !\mathcal{C} \) maps a list of lists of objects to the flatten list.



\ifdraft
\section*{PAGE LIMIT = 12 pages}
\fi

\bibliographystyle{IEEEtran}
\bibliography{jabref}

\iffalse
\section{Introduction}
This document is a model and instructions for \LaTeX.
Please observe the conference page limits. For more information about how to become an IEEE Conference author or how to write your paper, please visit   IEEE Conference Author Center website: https://conferences.ieeeauthorcenter.ieee.org/.

\subsection{Maintaining the Integrity of the Specifications}

The IEEEtran class file is used to format your paper and style the text. All margins, 
column widths, line spaces, and text fonts are prescribed; please do not 
alter them. You may note peculiarities. For example, the head margin
measures proportionately more than is customary. This measurement 
and others are deliberate, using specifications that anticipate your paper 
as one part of the entire proceedings, and not as an independent document. 
Please do not revise any of the current designations.

\section{Prepare Your Paper Before Styling}
Before you begin to format your paper, first write and save the content as a 
separate text file. Complete all content and organizational editing before 
formatting. Please note sections \ref{AA} to \ref{FAT} below for more information on 
proofreading, spelling and grammar.

Keep your text and graphic files separate until after the text has been 
formatted and styled. Do not number text heads---{\LaTeX} will do that 
for you.

\subsection{Abbreviations and Acronyms}\label{AA}
Define abbreviations and acronyms the first time they are used in the text, 
even after they have been defined in the abstract. Abbreviations such as 
IEEE, SI, MKS, CGS, ac, dc, and rms do not have to be defined. Do not use 
abbreviations in the title or heads unless they are unavoidable.

\subsection{Units}
\begin{itemize}
\item Use either SI (MKS) or CGS as primary units. (SI units are encouraged.) English units may be used as secondary units (in parentheses). An exception would be the use of English units as identifiers in trade, such as ``3.5-inch disk drive''.
\item Avoid combining SI and CGS units, such as current in amperes and magnetic field in oersteds. This often leads to confusion because equations do not balance dimensionally. If you must use mixed units, clearly state the units for each quantity that you use in an equation.
\item Do not mix complete spellings and abbreviations of units: ``Wb/m\textsuperscript{2}'' or ``webers per square meter'', not ``webers/m\textsuperscript{2}''. Spell out units when they appear in text: ``. . . a few henries'', not ``. . . a few H''.
\item Use a zero before decimal points: ``0.25'', not ``.25''. Use ``cm\textsuperscript{3}'', not ``cc''.)
\end{itemize}

\subsection{Equations}
Number equations consecutively. To make your 
equations more compact, you may use the solidus (~/~), the exp function, or 
appropriate exponents. Italicize Roman symbols for quantities and variables, 
but not Greek symbols. Use a long dash rather than a hyphen for a minus 
sign. Punctuate equations with commas or periods when they are part of a 
sentence, as in:
\begin{equation}
a+b=\gamma\label{eq}
\end{equation}

Be sure that the 
symbols in your equation have been defined before or immediately following 
the equation. Use ``\eqref{eq}'', not ``Eq.~\eqref{eq}'' or ``equation \eqref{eq}'', except at 
the beginning of a sentence: ``Equation \eqref{eq} is . . .''

\subsection{\LaTeX-Specific Advice}

Please use ``soft'' (e.g., \verb|\eqref{Eq}|) cross references instead
of ``hard'' references (e.g., \verb|(1)|). That will make it possible
to combine sections, add equations, or change the order of figures or
citations without having to go through the file line by line.

Please don't use the \verb|{eqnarray}| equation environment. Use
\verb|{align}| or \verb|{IEEEeqnarray}| instead. The \verb|{eqnarray}|
environment leaves unsightly spaces around relation symbols.

Please note that the \verb|{subequations}| environment in {\LaTeX}
will increment the main equation counter even when there are no
equation numbers displayed. If you forget that, you might write an
article in which the equation numbers skip from (17) to (20), causing
the copy editors to wonder if you've discovered a new method of
counting.

{\BibTeX} does not work by magic. It doesn't get the bibliographic
data from thin air but from .bib files. If you use {\BibTeX} to produce a
bibliography you must send the .bib files. 

{\LaTeX} can't read your mind. If you assign the same label to a
subsubsection and a table, you might find that Table I has been cross
referenced as Table IV-B3. 

{\LaTeX} does not have precognitive abilities. If you put a
\verb|\label| command before the command that updates the counter it's
supposed to be using, the label will pick up the last counter to be
cross referenced instead. In particular, a \verb|\label| command
should not go before the caption of a figure or a table.

Do not use \verb|\nonumber| inside the \verb|{array}| environment. It
will not stop equation numbers inside \verb|{array}| (there won't be
any anyway) and it might stop a wanted equation number in the
surrounding equation.

\subsection{Some Common Mistakes}\label{SCM}
\begin{itemize}
\item The word ``data'' is plural, not singular.
\item The subscript for the permeability of vacuum $\mu_{0}$, and other common scientific constants, is zero with subscript formatting, not a lowercase letter ``o''.
\item In American English, commas, semicolons, periods, question and exclamation marks are located within quotation marks only when a complete thought or name is cited, such as a title or full quotation. When quotation marks are used, instead of a bold or italic typeface, to highlight a word or phrase, punctuation should appear outside of the quotation marks. A parenthetical phrase or statement at the end of a sentence is punctuated outside of the closing parenthesis (like this). (A parenthetical sentence is punctuated within the parentheses.)
\item A graph within a graph is an ``inset'', not an ``insert''. The word alternatively is preferred to the word ``alternately'' (unless you really mean something that alternates).
\item Do not use the word ``essentially'' to mean ``approximately'' or ``effectively''.
\item In your paper title, if the words ``that uses'' can accurately replace the word ``using'', capitalize the ``u''; if not, keep using lower-cased.
\item Be aware of the different meanings of the homophones ``affect'' and ``effect'', ``complement'' and ``compliment'', ``discreet'' and ``discrete'', ``principal'' and ``principle''.
\item Do not confuse ``imply'' and ``infer''.
\item The prefix ``non'' is not a word; it should be joined to the word it modifies, usually without a hyphen.
\item There is no period after the ``et'' in the Latin abbreviation ``et al.''.
\item The abbreviation ``i.e.'' means ``that is'', and the abbreviation ``e.g.'' means ``for example''.
\end{itemize}
An excellent style manual for science writers is \cite{b7}.

\subsection{Authors and Affiliations}\label{AAA}
\textbf{The class file is designed for, but not limited to, six authors.} A 
minimum of one author is required for all conference articles. Author names 
should be listed starting from left to right and then moving down to the 
next line. This is the author sequence that will be used in future citations 
and by indexing services. Names should not be listed in columns nor group by 
affiliation. Please keep your affiliations as succinct as possible (for 
example, do not differentiate among departments of the same organization).

\subsection{Identify the Headings}\label{ITH}
Headings, or heads, are organizational devices that guide the reader through 
your paper. There are two types: component heads and text heads.

Component heads identify the different components of your paper and are not 
topically subordinate to each other. Examples include Acknowledgments and 
References and, for these, the correct style to use is ``Heading 5''. Use 
``figure caption'' for your Figure captions, and ``table head'' for your 
table title. Run-in heads, such as ``Abstract'', will require you to apply a 
style (in this case, italic) in addition to the style provided by the drop 
down menu to differentiate the head from the text.

Text heads organize the topics on a relational, hierarchical basis. For 
example, the paper title is the primary text head because all subsequent 
material relates and elaborates on this one topic. If there are two or more 
sub-topics, the next level head (uppercase Roman numerals) should be used 
and, conversely, if there are not at least two sub-topics, then no subheads 
should be introduced.

\subsection{Figures and Tables}\label{FAT}
\paragraph{Positioning Figures and Tables} Place figures and tables at the top and 
bottom of columns. Avoid placing them in the middle of columns. Large 
figures and tables may span across both columns. Figure captions should be 
below the figures; table heads should appear above the tables. Insert 
figures and tables after they are cited in the text. Use the abbreviation 
``Fig.~\ref{fig}'', even at the beginning of a sentence.

\begin{table}[htbp]
\caption{Table Type Styles}
\begin{center}
\begin{tabular}{|c|c|c|c|}
\hline
\textbf{Table}&\multicolumn{3}{|c|}{\textbf{Table Column Head}} \\
\cline{2-4} 
\textbf{Head} & \textbf{\textit{Table column subhead}}& \textbf{\textit{Subhead}}& \textbf{\textit{Subhead}} \\
\hline
copy& More table copy$^{\mathrm{a}}$& &  \\
\hline
\multicolumn{4}{l}{$^{\mathrm{a}}$Sample of a Table footnote.}
\end{tabular}
\label{tab1}
\end{center}
\end{table}

% \begin{figure}[htbp]
% \centerline{\includegraphics{fig1.png}}
% \caption{Example of a figure caption.}
% \label{fig}
% \end{figure}

Figure Labels: Use 8 point Times New Roman for Figure labels. Use words 
rather than symbols or abbreviations when writing Figure axis labels to 
avoid confusing the reader. As an example, write the quantity 
``Magnetization'', or ``Magnetization, M'', not just ``M''. If including 
units in the label, present them within parentheses. Do not label axes only 
with units. In the example, write ``Magnetization (A/m)'' or ``Magnetization 
\{A[m(1)]\}'', not just ``A/m''. Do not label axes with a ratio of 
quantities and units. For example, write ``Temperature (K)'', not 
``Temperature/K''.

\section*{Acknowledgment}

The preferred spelling of the word ``acknowledgment'' in America is without 
an ``e'' after the ``g''. Avoid the stilted expression ``one of us (R. B. 
G.) thanks $\ldots$''. Instead, try ``R. B. G. thanks$\ldots$''. Put sponsor 
acknowledgments in the unnumbered footnote on the first page.

\section*{References}

Please number citations consecutively within brackets \cite{b1}. The 
sentence punctuation follows the bracket \cite{b2}. Refer simply to the reference 
number, as in \cite{b3}---do not use ``Ref. \cite{b3}'' or ``reference \cite{b3}'' except at 
the beginning of a sentence: ``Reference \cite{b3} was the first $\ldots$''

Number footnotes separately in superscripts. Place the actual footnote at 
the bottom of the column in which it was cited. Do not put footnotes in the 
abstract or reference list. Use letters for table footnotes.

Unless there are six authors or more give all authors' names; do not use 
``et al.''. Papers that have not been published, even if they have been 
submitted for publication, should be cited as ``unpublished'' \cite{b4}. Papers 
that have been accepted for publication should be cited as ``in press'' \cite{b5}. 
Capitalize only the first word in a paper title, except for proper nouns and 
element symbols.

For papers published in translation journals, please give the English 
citation first, followed by the original foreign-language citation \cite{b6}.

\begin{thebibliography}{00}
\bibitem{b1} G. Eason, B. Noble, and I. N. Sneddon, ``On certain integrals of Lipschitz-Hankel type involving products of Bessel functions,'' Phil. Trans. Roy. Soc. London, vol. A247, pp. 529--551, April 1955.
\bibitem{b2} J. Clerk Maxwell, A Treatise on Electricity and Magnetism, 3rd ed., vol. 2. Oxford: Clarendon, 1892, pp.68--73.
\bibitem{b3} I. S. Jacobs and C. P. Bean, ``Fine particles, thin films and exchange anisotropy,'' in Magnetism, vol. III, G. T. Rado and H. Suhl, Eds. New York: Academic, 1963, pp. 271--350.
\bibitem{b4} K. Elissa, ``Title of paper if known,'' unpublished.
\bibitem{b5} R. Nicole, ``Title of paper with only first word capitalized,'' J. Name Stand. Abbrev., in press.
\bibitem{b6} Y. Yorozu, M. Hirano, K. Oka, and Y. Tagawa, ``Electron spectroscopy studies on magneto-optical media and plastic substrate interface,'' IEEE Transl. J. Magn. Japan, vol. 2, pp. 740--741, August 1987 [Digests 9th Annual Conf. Magnetics Japan, p. 301, 1982].
\bibitem{b7} M. Young, The Technical Writer's Handbook. Mill Valley, CA: University Science, 1989.
\bibitem{b8} D. P. Kingma and M. Welling, ``Auto-encoding variational Bayes,'' 2013, arXiv:1312.6114. [Online]. Available: https://arxiv.org/abs/1312.6114
\bibitem{b9} S. Liu, ``Wi-Fi Energy Detection Testbed (12MTC),'' 2023, gitHub repository. [Online]. Available: https://github.com/liustone99/Wi-Fi-Energy-Detection-Testbed-12MTC
\bibitem{b10} ``Treatment episode data set: discharges (TEDS-D): concatenated, 2006 to 2009.'' U.S. Department of Health and Human Services, Substance Abuse and Mental Health Services Administration, Office of Applied Studies, August, 2013, DOI:10.3886/ICPSR30122.v2
\bibitem{b11} K. Eves and J. Valasek, ``Adaptive control for singularly perturbed systems examples,'' Code Ocean, Aug. 2023. [Online]. Available: https://codeocean.com/capsule/4989235/tree
\end{thebibliography}

\vspace{12pt}
\color{red}
IEEE conference templates contain guidance text for composing and formatting conference papers. Please ensure that all template text is removed from your conference paper prior to submission to the conference. Failure to remove the template text from your paper may result in your paper not being published.
\fi

\end{document}
